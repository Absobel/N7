\section{Assemblage}
Dans cette section nous allons traiter le problème de l'assemblage.\\

Tout d'abord nous avons choisi le format \texttt{.lp} car le cas ici est très spécialisé et ne comporte que peu de données.\\

Le modèle est PLNE pour une semaine isolée avec des contraintes précises. En PL, on considère la semaine dans le contexte d'une succession, permettant des valeurs fractionnaires.

\subsection{Variables}
Les variables du problèmes sont les suivantes : \newline

\begin{itemize}

\item \makebox[2cm]{$p_{vc}$\hfill} : production de vélo cargo par semaine \newline
\item \makebox[2cm]{$p_{vs}$\hfill} : production de vélo standard par semaine \newline

\end{itemize}

\subsection{Fonction Objectif}
La fonction à maximiser dans le problème est la fonction :

$$max\ f(p_{vc},p_{vs})\ =\ max(700p_{vc}\ +\ 300 p_{vs})$$

\subsection{Contraintes}
Les contraintes de ce problème sont les suivantes:

\begin{itemize}

\item $ 0\ \le\ p_{vc}\ \le\ 700 $
\item $2,5 p_{vc}\ +\ p_{vs}\ \le\ 700 $
\item $\cfrac{(p_{vc} + p_{vs})11}{3}\ \le\ 60 $
\end{itemize}

\subsection{Domaine}
Les domaines des variables sont :

Dans le cas du modèle PLNE :

\begin{itemize}

\item $p_{vc} \in \mathbb{N} $
\item $p_{vs} \in \mathbb{N} $

\end{itemize}

Dans le cas du modèle PL :

\begin{itemize}

\item $p_{vc} \in \mathbb{R}_{+} $
\item $p_{vs} \in \mathbb{R}_{+} $

\end{itemize}


\subsection{Résultats}
Voici le résultat obtenu par GLPK pour ce problème

Pour le modèle PLNE :

\begin{lstlisting}
Problem:
Rows:       3
Columns:    2 (2 integer, 0 binary)
Non-zeros:  5
Status:     INTEGER OPTIMAL
Objective:  f = 438400 (MAXimum)

   No.   Row name        Activity     Lower bound   Upper bound
------ ------------    ------------- ------------- -------------
     1 HeureParSemaine
                               59.92                          60
     2 Parking                  1500                        1500
     3 ProdMaxCargo              232                         700

   No. Column name       Activity     Lower bound   Upper bound
------ ------------    ------------- ------------- -------------
     1 vc           *            232             0
     2 vs           *            920             0

Integer feasibility conditions:

KKT.PE: max.abs.err = 0.00e+00 on row 0
        max.rel.err = 0.00e+00 on row 0
        High quality

KKT.PB: max.abs.err = 0.00e+00 on row 0
        max.rel.err = 0.00e+00 on row 0
        High quality

End of output
\end{lstlisting}

Pour le modèle PL :

\begin{lstlisting}
    Problem:    
    Rows:       3
    Columns:    2
    Non-zeros:  5
    Status:     OPTIMAL
    Objective:  f = 438461.5385 (MAXimum)
    
       No.   Row name   St   Activity     Lower bound   Upper bound    Marginal
    ------ ------------ -- ------------- ------------- ------------- -------------
         1 HeureParSemaine
                        NU            60                          60       769.231 
         2 Parking      NU          1500                        1500       261.538 
         3 ProdMaxCargo B        230.769                         700 
    
       No. Column name  St   Activity     Lower bound   Upper bound    Marginal
    ------ ------------ -- ------------- ------------- ------------- -------------
         1 vc           B        230.769             0               
         2 vs           B        923.077             0               
    
    Karush-Kuhn-Tucker optimality conditions:
    
    KKT.PE: max.abs.err = 7.11e-15 on row 1
            max.rel.err = 5.87e-17 on row 1
            High quality
    
    KKT.PB: max.abs.err = 0.00e+00 on row 0
            max.rel.err = 0.00e+00 on row 0
            High quality
    
    KKT.DE: max.abs.err = 0.00e+00 on column 0
            max.rel.err = 0.00e+00 on column 0
            High quality
    
    KKT.DB: max.abs.err = 0.00e+00 on row 0
            max.rel.err = 0.00e+00 on row 0
            High quality
    
    End of output
\end{lstlisting}  

On voit que le maximum est atteint pour une valeur de $f=438400$ avec comme conditions :
\begin{itemize}
\item $\text{HeureParSemaine} = 59.92$
\item $\text{Parking} = 1500$
\item $\text{ProdMaxCargo} = 232$
\item 232 vélos de type cargo vendus
\item 920 vélos de type standard vendus
\end{itemize}
Respectivement, pour le modèle PL :
\begin{itemize}
\item $\text{HeureParSemaine} = 60$
\item $\text{Parking} = 1500$
\item $\text{ProdMaxCargo} = 230.769$
\item 230.769 vélos de type cargo vendus
\item 923.077 vélos de type standard vendus
\end{itemize}

On peut à présent se demander si ces solutions sont réalistes.
\\ On remarque que les bornes sur les contraintes sont respectées.


\section{Affectation avec prise en compte des préférences}
Dans cette section nous allons traiter le problème de l'affectation avec prise en compte des préférences.\\
Tout d'abord nous avons choisi le format \texttt{.mod} et \texttt{.dat} car on souhaite pouvoir généraliser ce cas à $N$ personnes.\\
De plus ici les données sont importantes et variables (matrice de de taille $N^2$). \\
Ainsi il est plus judicieux d'utiliser un format \texttt{dat} pour pouvoir facilement modifier/importer les données.

\subsection{Variables}
Les variables du problèmes sont les suivantes : \newline

\begin{itemize}

\item \makebox[2cm]{$A_{tp}$\hfill} : Matrice tâche-préférences \newline

\end{itemize}

\subsection{Fonction Objectif}
La fonction à maximiser dans le problème est la fonction :

$$max\ \sum_{i=1}^{n}\sum\limits_{j=1}^{n}A_{tp}(i,j)*c(i,j)$$

\subsection{Contraintes}
Les contraintes de ce problème sont les suivantes:

\begin{itemize}

\item $\forall j \in [1,..,n],\ \sum\limits_{i=0}^{n}A_{tp}(i,j) = 1$
\item $\forall i \in [1,..,n],\ \sum\limits_{j=0}^{n}A_{tp}(i,j) = 1$

\end{itemize}

\subsection{Domaine}
Les domaines des variables sont :
\begin{itemize}

\item $A_{tp} \in \mathcal{M}_n(\mathbb{B}) $

\end{itemize}

\subsection{Résultats}
Voici les résultats obtenus par GLPK pour ce problème.


\subsubsection{Exemple 1}
Voici la matrice de contraintes de l'exemple 1:
\[
\begin{bmatrix}
    4 & 7 & 9 \\
    9 & 8 & 3 \\
    2 & 1 & 2 \\
\end{bmatrix}
\]

et la maximisation est atteinte pour la matrice binaire suivante
\[
\begin{bmatrix}
    0 & 0 & 1 \\
    1 & 0 & 0 \\
    0 & 1 & 0 \\
\end{bmatrix}
\]
On voit que la contrainte sur les lignes et la contrainte sur les colonnes est respectées.
Cette solution est donc une solution cohérente

\subsubsection{Exemple 2}
Voici la matrice de contraintes de l'exemple 2:

\[
\begin{bmatrix}
    4 & 7 & 8 & 2 & 10 \\
    1 & 5 & 10 & 9 & 7 \\
    10 & 7 & 7 & 5 & 8 \\
    10 & 2 & 2 & 7 & 4 \\
    2 & 9 & 6 & 4 & 9 \\
\end{bmatrix}
\]


et la maximisation est atteinte pour la matrice binaire suivante
\[
\begin{bmatrix}
    0 & 0 & 0 & 0 & 1 \\
    0 & 0 & 1 & 0 & 0 \\
    1 & 0 & 0 & 0 & 0 \\
    0 & 0 & 0 & 1 & 0 \\
    0 & 1 & 0 & 0 & 0 \\
\end{bmatrix}
\]

On voit que la contrainte sur les lignes et la contrainte sur les colonnes est respectées.
Cette solution est donc une solution cohérente.


\subsubsection{Exemple 2}
Voici la matrice de contraintes de l'exemple 2:

\[
\begin{bmatrix}
    10 & 7 & 4 & 1 & 10 & 10 & 9 & 1 & 4 & 7 \\
    8 & 9 & 1 & 6 & 2 & 4 & 2 & 4 & 1 & 8 \\
    3 & 5 & 3 & 1 & 1 & 3 & 9 & 8 & 4 & 6 \\
    4 & 3 & 8 & 7 & 2 & 3 & 4 & 7 & 10 & 9 \\
    6 & 10 & 6 & 3 & 6 & 7 & 3 & 1 & 5 & 6 \\
    7 & 10 & 2 & 10 & 8 & 10 & 4 & 6 & 10 & 6 \\
    7 & 6 & 5 & 2 & 7 & 2 & 10 & 5 & 5 & 2 \\
    9 & 10 & 4 & 6 & 4 & 4 & 2 & 9 & 9 & 8 \\
    1 & 6 & 7 & 4 & 1 & 8 & 6 & 9 & 9 & 1 \\
    1 & 10 & 10 & 4 & 2 & 1 & 8 & 3 & 6 & 10 \\
\end{bmatrix}
\]



et la maximisation est atteinte pour la matrice binaire suivante
\[
\begin{bmatrix}
    0 & 0 & 0 & 0 & 1 & 0 & 0 & 0 & 0 & 0 \\
    0 & 0 & 0 & 0 & 0 & 0 & 0 & 0 & 0 & 1 \\
    0 & 0 & 0 & 0 & 0 & 0 & 0 & 1 & 0 & 0 \\
    0 & 0 & 0 & 0 & 0 & 0 & 0 & 0 & 1 & 0 \\
    0 & 1 & 0 & 0 & 0 & 0 & 0 & 0 & 0 & 0 \\
    0 & 0 & 0 & 1 & 0 & 0 & 0 & 0 & 0 & 0 \\
    0 & 0 & 0 & 0 & 0 & 0 & 1 & 0 & 0 & 0 \\
    1 & 0 & 0 & 0 & 0 & 0 & 0 & 0 & 0 & 0 \\
    0 & 0 & 0 & 0 & 0 & 1 & 0 & 0 & 0 & 0 \\
    0 & 0 & 1 & 0 & 0 & 0 & 0 & 0 & 0 & 0 \\
\end{bmatrix}
\]

On voit que la contrainte sur les lignes et la contrainte sur les colonnes est respectées.
Cette solution est donc une solution cohérente.


