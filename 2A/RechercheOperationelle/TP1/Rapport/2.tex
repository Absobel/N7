

\section{Cas particulier 1.1}

\subsection{Variables}

Les variables du problème sont les suivantes:

\begin{itemize}
    \item $q_{mfd}$ : Quantité de produit $f$ stockée dans le magasin $m$ pour la demande $d$.
\end{itemize}

\subsection{Fonction Objectif}

Le problème vise à minimiser le coût total, défini par la fonction objectif suivante :

\[
min \sum_{m=1}^{nbM} \sum_{f=1}^{nbF} \sum_{d=1}^{nbD} q_{mfd} \cdot COUTS_{mf}
\]

\subsection{Contraintes}

Les contraintes du problème sont les suivantes :

\begin{itemize}
    \item Demande Respectée : $\forall d \in [1, \ldots, nbD], \forall f \in [1, \ldots, nbF], \text{DEMANDES}_{df} \leq \sum\limits_{m=1}^{nbM} q_{mfd}$
    \item Stock Respecté : $\forall m \in [1, \ldots, nbM], \forall f \in [1, \ldots, nbF], \sum\limits_{d=1}^{nbD} q_{mfd} \leq \text{STOCK}_{mf}$
\end{itemize}

\subsection{Domaine}

Les domaines des variables sont les suivants :

\begin{itemize}
\item \[\forall m \in [1, \ldots, nbM], \forall f \in [1, \ldots, nbF], \forall d \in [1, \ldots, nbD] ,\ q_{mfd} \geq 0
\]
\end{itemize}

\subsection{Conclusion}

Voici les données du problème posé :

\begin{itemize}
\item le nombre de fluide est égal à 2
\item le nombre de magasin est égal à 3
\item le nombre de demandes est égal à 2
\item Le tableau de stock de fluide \\

\begin{tabular}{|c|c|c|}
\hline
\textbf{} & \textbf{Fluide 1} & \textbf{Fluide 2} \\
\hline
\textbf{Magasin 1} & 2.5 & 1 \\
\textbf{Magasin 2} & 1 & 2 \\
\textbf{Magasin 3} & 2 & 1 \\
\hline
\end{tabular}

\item le tableau des demandes de fluides \\

\begin{tabular}{|c|c|c|}
\hline
\textbf{} & \textbf{Fluide 1} & \textbf{Fluide 2} \\
\hline
\textbf{Demande 1} & 2 & 0 \\
\textbf{Demande 2} & 1 & 3 \\
\hline
\end{tabular}


\item le tableau des coûts unitaires par magasin \\

\begin{tabular}{|c|c|c|}
\hline
\textbf{} & \textbf{Fluide 1} & \textbf{Fluide 2} \\
\hline
\textbf{Magasin 1} & 1 & 1 \\
\textbf{Magasin 2} & 2 & 3 \\
\textbf{Magasin 3} & 3 & 2 \\
\hline
\end{tabular}


\end{itemize}

Avec ces données, nous obtenons le résutat suivant :


\begin{lstlisting}
Problem:    ECommerceCP11
Rows:       11
Columns:    12
Non-zeros:  36
Status:     OPTIMAL
Objective:  coutTotal = 9.5 (MINimum)

   No.   Row name   St   Activity     Lower bound   Upper bound    Marginal
------ ------------ -- ------------- ------------- ------------- -------------
     1 coutTotal    B            9.5
     2 demandeRespectee[1,1]
                    NU            -2                          -2            -2
     3 demandeRespectee[1,2]
                    B              0                          -0
     4 demandeRespectee[2,1]
                    NU            -1                          -1            -2
     5 demandeRespectee[2,2]
                    NU            -3                          -3            -3
     6 stockRespecte[1,1]
                    NU           2.5                         2.5            -1
     7 stockRespecte[1,2]
                    NU             1                           1            -2
     8 stockRespecte[2,1]
                    B            0.5                           1
     9 stockRespecte[2,2]
                    B              1                           2
    10 stockRespecte[3,1]
                    B              0                           2
    11 stockRespecte[3,2]
                    NU             1                           1            -1

   No. Column name  St   Activity     Lower bound   Upper bound    Marginal
------ ------------ -- ------------- ------------- ------------- -------------
     1 q[1,1,1]     B              2             0
     2 q[1,1,2]     B            0.5             0
     3 q[1,2,1]     NL             0             0                           3
     4 q[1,2,2]     B              1             0
     5 q[2,1,1]     NL             0             0                       < eps
     6 q[2,1,2]     B            0.5             0
     7 q[2,2,1]     NL             0             0                           3
     8 q[2,2,2]     B              1             0
     9 q[3,1,1]     NL             0             0                           1
    10 q[3,1,2]     NL             0             0                           1
    11 q[3,2,1]     NL             0             0                           3
    12 q[3,2,2]     B              1             0

Karush-Kuhn-Tucker optimality conditions:

KKT.PE: max.abs.err = 0.00e+00 on row 0
        max.rel.err = 0.00e+00 on row 0
        High quality

KKT.PB: max.abs.err = 0.00e+00 on row 0
        max.rel.err = 0.00e+00 on row 0
        High quality

KKT.DE: max.abs.err = 0.00e+00 on column 0
        max.rel.err = 0.00e+00 on column 0
        High quality

KKT.DB: max.abs.err = 0.00e+00 on row 0
        max.rel.err = 0.00e+00 on row 0
        High quality

End of output

\end{lstlisting}

Donc GPLK propose le scénario suivant :
\begin{enumerate}

\item Pour la demande 1, on va chercher dans le magasin 1 2 unités de $fluide_1$
\item Pour la demande 2, on va chercher dans le magasin 1 0.5 unité de $fluide_1$
\item Pour la demande 2, on va chercher dans le magasin 1 1 unité de $fluide_2$
\item Pour la demande 2, on va chercher dans le magasin 2 0.5 unité de $fluide_1$
\item Pour la demande 2, on va chercher dans le magasin 2 2 unités de $fluide_2$
\item Pour la demande 2, on va chercher dans le magasin 3 1 unité de $fluide_2$
\end{enumerate}

Ce qui donne les tableaux itérés suivants:
\begin{enumerate}
\item
\begin{table}[H]
\centering
\begin{tabular}{c c}

\begin{tabular}{|c|c|c|}
\hline
\textbf{} & \textbf{F1} & \textbf{F2} \\
\hline
\textbf{M1} & 0.5 & 1 \\
\textbf{M2} & 1 & 2 \\
\textbf{M3} & 2 & 1 \\
\hline
\end{tabular}
&
\begin{tabular}{|c|c|c|}
\hline
\textbf{} & \textbf{F1} & \textbf{F2} \\
\hline
\textbf{D1} & 0 & 0 \\
\textbf{D2} & 1 & 3 \\
\hline
\end{tabular}

\end{tabular}
\end{table}

\item
\begin{table}[H]
\centering
\begin{tabular}{c c}

\begin{tabular}{|c|c|c|}
\hline
\textbf{} & \textbf{F1} & \textbf{F2} \\
\hline
\textbf{M1} & 0 & 1 \\
\textbf{M2} & 1 & 2 \\
\textbf{M3} & 2 & 1 \\
\hline
\end{tabular}
&
\begin{tabular}{|c|c|c|}
\hline
\textbf{} & \textbf{F1} & \textbf{F2} \\
\hline
\textbf{D1} & 0 & 0 \\
\textbf{D2} & 0.5 & 3 \\
\hline
\end{tabular}

\end{tabular}
\end{table}

\item
\begin{table}[h]
\centering
\begin{tabular}{c c}

\begin{tabular}{|c|c|c|}
\hline
\textbf{} & \textbf{F1} & \textbf{F2} \\
\hline
\textbf{M1} & 0 & 0 \\
\textbf{M2} & 1 & 2 \\
\textbf{M3} & 2 & 1 \\
\hline
\end{tabular}
&
\begin{tabular}{|c|c|c|}
\hline
\textbf{} & \textbf{F1} & \textbf{F2} \\
\hline
\textbf{D1} & 0 & 0 \\
\textbf{D2} & 0.5 & 3 \\
\hline
\end{tabular}

\end{tabular}
\end{table}

\item
\begin{table}[H]
\centering
\begin{tabular}{c c}

\begin{tabular}{|c|c|c|}
\hline
\textbf{} & \textbf{F1} & \textbf{F2} \\
\hline
\textbf{M1} & 0 & 0 \\
\textbf{M2} & 0.5 & 2 \\
\textbf{M3} & 2 & 1 \\
\hline
\end{tabular}
&
\begin{tabular}{|c|c|c|}
\hline
\textbf{} & \textbf{F1} & \textbf{F2} \\
\hline
\textbf{D1} & 0 & 0 \\
\textbf{D2} & 0 & 3 \\
\hline
\end{tabular}

\end{tabular}
\end{table}

\item
\begin{table}[H]
\centering
\begin{tabular}{c c}

\begin{tabular}{|c|c|c|}
\hline
\textbf{} & \textbf{F1} & \textbf{F2} \\
\hline
\textbf{M1} & 0 & 0 \\
\textbf{M2} & 0.5 & 0 \\
\textbf{M3} & 2 & 1 \\
\hline
\end{tabular}
&
\begin{tabular}{|c|c|c|}
\hline
\textbf{} & \textbf{F1} & \textbf{F2} \\
\hline
\textbf{D1} & 0 & 0 \\
\textbf{D2} & 0 & 1 \\
\hline
\end{tabular}

\end{tabular}
\end{table}

\item
\begin{table}[H]
\centering
\begin{tabular}{c c}

\begin{tabular}{|c|c|c|}
\hline
\textbf{} & \textbf{F1} & \textbf{F2} \\
\hline
\textbf{M1} & 0 & 0 \\
\textbf{M2} & 0.5 & 0 \\
\textbf{M3} & 2 & 0 \\
\hline
\end{tabular}
&
\begin{tabular}{|c|c|c|}
\hline
\textbf{} & \textbf{F1} & \textbf{F2} \\
\hline
\textbf{D1} & 0 & 0 \\
\textbf{D2} & 0 & 0 \\
\hline
\end{tabular}

\end{tabular}
\end{table}
\end{enumerate}

Cela montre que la solution est cohérente étant donné que chaque état de la solution respecte les contraintes données.



