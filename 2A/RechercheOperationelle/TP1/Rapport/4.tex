\section{Cas particulier 2}
Ce problème correspond au problème théorique du voyageur car il s'agit de trouver le chemin le plus court pour livrer les colis au bon endroit.



\subsection{Variables}
Dans ce problème nous avons identifié 2 variables :
\begin{itemize}
\item $Y$ : Matrice des arcs entre les clients. Le coefficient $c(i,j)$ indique  dans le cas où il vaut 1 que le livreur part du client $i$ et se dirige vers le client $j$
\item $T$ : Vecteur de l'ordre de livraison
\end{itemize}

\subsection{Fonction Objectif}

La fonction à minimiser dans le problème est la fonction :

$$ min \sum_{i=1}^{n}\sum_{j=1}^{n}  DISTANCE_{ij} * Y_{ij}$$

\subsection{Contraintes}

\text{Assure qu'il y ait un seul et unique 1 par ligne:\\}
$$ \quad \sum\limits_{j=1}^{N} Y[i,j] = 1, \quad \forall i \in \{1,2,\ldots,N\}$$ \\

\text{Assure qu'il y ait un seul et unique 1 par colonne:\\}
$$ \quad \sum\limits_{i=1}^{N} Y[i,j] = 1, \quad \forall j \in \{1,2,\ldots,N\}$$ \\
\text{Assure qu'un client ne peut pas se visiter après s'être visité :\\}
$$\quad Y[i,i] = 0, \quad \forall i \in \{1,2,\ldots,N\}$$ \\
\text{Assure qu'il n'y ait pas de sous-cycle au sein du graphe representant les clients et leur lirvraison:\\}
$$ \quad T[j] + (1 - Y[i,j]) \cdot M \geq T[i] + 1, \quad \forall i \in \{1,2,\ldots,N\}, \forall j \in \{2,3,\ldots,N\}$$\\
\text{Assure que l'ordre est positif:\\}
$$ \quad T[i] \geq 0, \quad \forall i \in \{1,2,\ldots,N\}$$



\subsection{Domaine}

\begin{itemize}
\item $Y \in \mathcal{M}_n(\mathbb{B}) $
\item $T \in \mathbb{R}^{n}$
\end{itemize}

\subsection{Conclusion}
Avec les paramètres précedent nous obtenons la solution suivante avec GLPK :
\begin{lstlisting}
Problem:    ECommerceCP2
Rows:       55
Columns:    42 (42 integer, 36 binary)
Non-zeros:  200
Status:     INTEGER OPTIMAL
Objective:  Distance = 22 (MINimum)

   No.   Row name        Activity     Lower bound   Upper bound
------ ------------    ------------- ------------- -------------
     1 Distance                   22
     2 RegleUn[1]                  1             1             =
     3 RegleUn[2]                  1             1             =
     4 RegleUn[3]                  1             1             =
     5 RegleUn[4]                  1             1             =
     6 RegleUn[5]                  1             1             =
     7 RegleUn[6]                  1             1             =
     8 RegleDeux[1]                1             1             =
     9 RegleDeux[2]                1             1             =
    10 RegleDeux[3]                1             1             =
    11 RegleDeux[4]                1             1             =
    12 RegleDeux[5]                1             1             =
    13 RegleDeux[6]                1             1             =
    14 RegleTrois[1]
                                   0            -0             =
    15 RegleTrois[2]
                                   0            -0             =
    16 RegleTrois[3]
                                   0            -0             =
    17 RegleTrois[4]
                                   0            -0             =
    18 RegleTrois[5]
                                   0            -0             =
    19 RegleTrois[6]
                                   0            -0             =
    20 RegleQuatre[1,2]
                                -999          -999
    21 RegleQuatre[1,3]
                                   5          -999
    22 RegleQuatre[1,4]
                                   2          -999
    23 RegleQuatre[1,5]
                                   3          -999
    24 RegleQuatre[1,6]
                                   4          -999
    25 RegleQuatre[2,2]
                                   0          -999
    26 RegleQuatre[2,3]
                                   4          -999
    27 RegleQuatre[2,4]
                                -999          -999
    28 RegleQuatre[2,5]
                                   2          -999
    29 RegleQuatre[2,6]
                                   3          -999
    30 RegleQuatre[3,2]
                                  -4          -999
    31 RegleQuatre[3,3]
                                   0          -999
    32 RegleQuatre[3,4]
                                  -3          -999
    33 RegleQuatre[3,5]
                                  -2          -999
    34 RegleQuatre[3,6]
                                  -1          -999
    35 RegleQuatre[4,2]
                                  -1          -999
    36 RegleQuatre[4,3]
                                   3          -999
    37 RegleQuatre[4,4]
                                   0          -999
    38 RegleQuatre[4,5]
                                -999          -999
    39 RegleQuatre[4,6]
                                   2          -999
    40 RegleQuatre[5,2]
                                  -2          -999
    41 RegleQuatre[5,3]
                                   2          -999
    42 RegleQuatre[5,4]
                                  -1          -999
    43 RegleQuatre[5,5]
                                   0          -999
    44 RegleQuatre[5,6]
                                -999          -999
    45 RegleQuatre[6,2]
                                  -3          -999
    46 RegleQuatre[6,3]
                                -999          -999
    47 RegleQuatre[6,4]
                                  -2          -999
    48 RegleQuatre[6,5]
                                  -1          -999
    49 RegleQuatre[6,6]
                                   0          -999
    50 Reglex[1]                   0            -0
    51 Reglex[2]                   1            -0
    52 Reglex[3]                   5            -0
    53 Reglex[4]                   2            -0
    54 Reglex[5]                   3            -0
    55 Reglex[6]                   4            -0

   No. Column name       Activity     Lower bound   Upper bound
------ ------------    ------------- ------------- -------------
     1 Y[1,1]       *              0             0             1
     2 Y[1,2]       *              1             0             1
     3 Y[1,3]       *              0             0             1
     4 Y[1,4]       *              0             0             1
     5 Y[1,5]       *              0             0             1
     6 Y[1,6]       *              0             0             1
     7 Y[2,1]       *              0             0             1
     8 Y[2,2]       *              0             0             1
     9 Y[2,3]       *              0             0             1
    10 Y[2,4]       *              1             0             1
    11 Y[2,5]       *              0             0             1
    12 Y[2,6]       *              0             0             1
    13 Y[3,1]       *              1             0             1
    14 Y[3,2]       *              0             0             1
    15 Y[3,3]       *              0             0             1
    16 Y[3,4]       *              0             0             1
    17 Y[3,5]       *              0             0             1
    18 Y[3,6]       *              0             0             1
    19 Y[4,1]       *              0             0             1
    20 Y[4,2]       *              0             0             1
    21 Y[4,3]       *              0             0             1
    22 Y[4,4]       *              0             0             1
    23 Y[4,5]       *              1             0             1
    24 Y[4,6]       *              0             0             1
    25 Y[5,1]       *              0             0             1
    26 Y[5,2]       *              0             0             1
    27 Y[5,3]       *              0             0             1
    28 Y[5,4]       *              0             0             1
    29 Y[5,5]       *              0             0             1
    30 Y[5,6]       *              1             0             1
    31 Y[6,1]       *              0             0             1
    32 Y[6,2]       *              0             0             1
    33 Y[6,3]       *              1             0             1
    34 Y[6,4]       *              0             0             1
    35 Y[6,5]       *              0             0             1
    36 Y[6,6]       *              0             0             1
    37 T[2]         *              1
    38 T[1]         *              0
    39 T[3]         *              5
    40 T[4]         *              2
    41 T[5]         *              3
    42 T[6]         *              4

Integer feasibility conditions:

KKT.PE: max.abs.err = 0.00e+00 on row 0
        max.rel.err = 0.00e+00 on row 0
        High quality

KKT.PB: max.abs.err = 0.00e+00 on row 0
        max.rel.err = 0.00e+00 on row 0
        High quality

End of output

\end{lstlisting}

Vérifions que cette solution est cohérente.
La matrice $Y$ nous indique que le livreur emprunte le chemin suivant : \\
$$ALPHA \rightarrow client1 \rightarrow client3 \rightarrow client4 \rightarrow client5 \rightarrow ALPHA $$
Cette solution est cohérente pour les raisons suivantes :
\begin{itemize}
\item En terme de graphe c'est un cycle hamiltonien. On passe une et une seule fois chez chaque client.
\item Chaque ligne et colonne ne possède qu'un seul 1 et le reste de 0.
\end{itemize}



