
\documentclass[12pt]{article}
\usepackage{graphicx} % Required for inserting images
\usepackage[T1]{fontenc}
\usepackage{lmodern}
\usepackage{array}
\usepackage{a4wide}
\usepackage{amssymb}
\usepackage{blindtext}
\usepackage{titlesec}
\usepackage{color}
\usepackage{babel}
\usepackage{listings}
\usepackage{amsmath}
\usepackage{courier}
\usepackage{subfigure}
\usepackage{float}
\usepackage{xcolor}
\usepackage{amsfonts}
\setcounter{section}{0}
\setcounter{secnumdepth}{0}



\definecolor{codegreen}{rgb}{0,0.6,0}
\definecolor{codegray}{rgb}{0.5,0.5,0.5}
\definecolor{codepurple}{rgb}{0.58,0,0.82}
\definecolor{backcolour}{rgb}{0.95,0.95,0.92}

\lstdefinestyle{mystyle}{
    backgroundcolor=\color{backcolour},
    commentstyle=\color{codegreen},
    keywordstyle=\color{magenta},
    numberstyle=\tiny\color{codegray},
    stringstyle=\color{codepurple},
    basicstyle=\ttfamily\footnotesize,
    breakatwhitespace=false,
    breaklines=true,
    captionpos=b,
    keepspaces=true,
    numbers=left,
    numbersep=5pt,
    showspaces=false,
    showstringspaces=false,
    showtabs=false,
    tabsize=2
}


\lstset{style=mystyle}

\begin{document}




\title{\vspace{4cm} \textbf{Rapport recherche opérationnelle TP \\ Sujet 1}}
\author{Absobel, CORBELLARI Nolan}
\date{\vspace{7cm} Département Sciences du Numérique - Deuxième année \\
2023-2024}

\maketitle

\newpage
\pagebreak
\vfill

\newpage
\tableofcontents
\newpage

\section{Introduction et spécification des choix}
Nous allons dans ce rapport traiter les différents problèmes vus en TD. GLPK nous fournis une solution à partir des contraintes que nous lui fournissons,
cependant nous ne pouvons pas savoir à la main si la solution est optimale ou non.
Nous avons donc fait le choix de nous appuyer sur le déroulement de la solution ou des propriétés tel que les bornes inférieures et supérieures afin de justifier qu'une solution est cohérente.

\section {Questions préliminaires}
\begin{enumerate}
    \item
          À chaque noeud, le solveur va résoudre un problème d'optimisation sous contraintes.
          Chaque valeur de  $x(i)$ est à priori binaire et indique si oui ou non l'objet est inclus dans le sac.
          Cependant le solveur peut trouver une solution comportant un ou plusieurs $x(i)$ dont la valeur n'est pas binaire mais flottante.
          C'est sur ce critère que l'algorithme se base pour créer les branches à partir d'un noeud.
          Lorsqu'un résultat obtenu par un solveur sur un noeud de profondeur $i$ comporte une ou plusieurs valeurs non booléenne (notons les $x_{\lambda1},x_{\lambda2},...,x_{\lambda m})$ dans l'ordre lexicographique),
          l'algorithme selectionne $x_{\lambda_1}$ car c'est le premier dans l'ordre lexicographique et
          il créée deux noeuds de profondeur $i+1$ et une arrête possède la condition  $x_{\lambda_1} = 1$ et l'autre possède la condition $x_{\lambda_1} = 0$

    \item
          L'algorithme calcule la borne supérieur et inférieure de la manière suivante :

          \begin{itemize}
              \item borne supérieur :
                    $$ \forall i \in  [1..n], BorneSup(x[i]) = min(x[i],1)$$
              \item borne inférieure :
                    $$ \forall i \in  [1..n], BorneInf(x[i]) = max(x[i],0)$$
          \end{itemize}

    \item
          L'algorithme calcule la TA, la TO et la TR de la manière suivante :

          \begin{itemize}
              \item TA : Le problème d'optimisation sous contraintes ne possède pas de solution.
              \item TO : Le problème d'optimisation sous contraintes possède une solution moins intéressante que celle déjà enregistrée.
              \item TR : Le problème d'optimisation sous contraintes possède une solution plus intéressante que celle déjà enregistrée.
          \end{itemize}



    \item La stratégie de parcours est un parcours en profondeur de l'arbre crée en prenant d'abord le fils gauche plutot que le fils droit.

\end{enumerate}
 \newpage


\section{Cas particulier 1.1}

\subsection{Variables}

Les variables du problème sont les suivantes:

\begin{itemize}
    \item $q_{mfd}$ : Quantité de produit $f$ stockée dans le magasin $m$ pour la demande $d$.
\end{itemize}

\subsection{Fonction Objectif}

Le problème vise à minimiser le coût total, défini par la fonction objectif suivante :

\[
min \sum_{m=1}^{nbM} \sum_{f=1}^{nbF} \sum_{d=1}^{nbD} q_{mfd} \cdot COUTS_{mf}
\]

\subsection{Contraintes}

Les contraintes du problème sont les suivantes :

\begin{itemize}
    \item Demande Respectée : $\forall d \in [1, \ldots, nbD], \forall f \in [1, \ldots, nbF], \text{DEMANDES}_{df} \leq \sum\limits_{m=1}^{nbM} q_{mfd}$
    \item Stock Respecté : $\forall m \in [1, \ldots, nbM], \forall f \in [1, \ldots, nbF], \sum\limits_{d=1}^{nbD} q_{mfd} \leq \text{STOCK}_{mf}$
\end{itemize}

\subsection{Domaine}

Les domaines des variables sont les suivants :

\begin{itemize}
\item \[\forall m \in [1, \ldots, nbM], \forall f \in [1, \ldots, nbF], \forall d \in [1, \ldots, nbD] ,\ q_{mfd} \geq 0
\]
\end{itemize}

\subsection{Conclusion}

Voici les données du problème posé :

\begin{itemize}
\item le nombre de fluide est égal à 2
\item le nombre de magasin est égal à 3
\item le nombre de demandes est égal à 2
\item Le tableau de stock de fluide \\

\begin{tabular}{|c|c|c|}
\hline
\textbf{} & \textbf{Fluide 1} & \textbf{Fluide 2} \\
\hline
\textbf{Magasin 1} & 2.5 & 1 \\
\textbf{Magasin 2} & 1 & 2 \\
\textbf{Magasin 3} & 2 & 1 \\
\hline
\end{tabular}

\item le tableau des demandes de fluides \\

\begin{tabular}{|c|c|c|}
\hline
\textbf{} & \textbf{Fluide 1} & \textbf{Fluide 2} \\
\hline
\textbf{Demande 1} & 2 & 0 \\
\textbf{Demande 2} & 1 & 3 \\
\hline
\end{tabular}


\item le tableau des coûts unitaires par magasin \\

\begin{tabular}{|c|c|c|}
\hline
\textbf{} & \textbf{Fluide 1} & \textbf{Fluide 2} \\
\hline
\textbf{Magasin 1} & 1 & 1 \\
\textbf{Magasin 2} & 2 & 3 \\
\textbf{Magasin 3} & 3 & 2 \\
\hline
\end{tabular}


\end{itemize}

Avec ces données, nous obtenons le résutat suivant :


\begin{lstlisting}
Problem:    ECommerceCP11
Rows:       11
Columns:    12
Non-zeros:  36
Status:     OPTIMAL
Objective:  coutTotal = 9.5 (MINimum)

   No.   Row name   St   Activity     Lower bound   Upper bound    Marginal
------ ------------ -- ------------- ------------- ------------- -------------
     1 coutTotal    B            9.5
     2 demandeRespectee[1,1]
                    NU            -2                          -2            -2
     3 demandeRespectee[1,2]
                    B              0                          -0
     4 demandeRespectee[2,1]
                    NU            -1                          -1            -2
     5 demandeRespectee[2,2]
                    NU            -3                          -3            -3
     6 stockRespecte[1,1]
                    NU           2.5                         2.5            -1
     7 stockRespecte[1,2]
                    NU             1                           1            -2
     8 stockRespecte[2,1]
                    B            0.5                           1
     9 stockRespecte[2,2]
                    B              1                           2
    10 stockRespecte[3,1]
                    B              0                           2
    11 stockRespecte[3,2]
                    NU             1                           1            -1

   No. Column name  St   Activity     Lower bound   Upper bound    Marginal
------ ------------ -- ------------- ------------- ------------- -------------
     1 q[1,1,1]     B              2             0
     2 q[1,1,2]     B            0.5             0
     3 q[1,2,1]     NL             0             0                           3
     4 q[1,2,2]     B              1             0
     5 q[2,1,1]     NL             0             0                       < eps
     6 q[2,1,2]     B            0.5             0
     7 q[2,2,1]     NL             0             0                           3
     8 q[2,2,2]     B              1             0
     9 q[3,1,1]     NL             0             0                           1
    10 q[3,1,2]     NL             0             0                           1
    11 q[3,2,1]     NL             0             0                           3
    12 q[3,2,2]     B              1             0

Karush-Kuhn-Tucker optimality conditions:

KKT.PE: max.abs.err = 0.00e+00 on row 0
        max.rel.err = 0.00e+00 on row 0
        High quality

KKT.PB: max.abs.err = 0.00e+00 on row 0
        max.rel.err = 0.00e+00 on row 0
        High quality

KKT.DE: max.abs.err = 0.00e+00 on column 0
        max.rel.err = 0.00e+00 on column 0
        High quality

KKT.DB: max.abs.err = 0.00e+00 on row 0
        max.rel.err = 0.00e+00 on row 0
        High quality

End of output

\end{lstlisting}

Donc GPLK propose le scénario suivant :
\begin{enumerate}

\item Pour la demande 1, on va chercher dans le magasin 1 2 unités de $fluide_1$
\item Pour la demande 2, on va chercher dans le magasin 1 0.5 unité de $fluide_1$
\item Pour la demande 2, on va chercher dans le magasin 1 1 unité de $fluide_2$
\item Pour la demande 2, on va chercher dans le magasin 2 0.5 unité de $fluide_1$
\item Pour la demande 2, on va chercher dans le magasin 2 2 unités de $fluide_2$
\item Pour la demande 2, on va chercher dans le magasin 3 1 unité de $fluide_2$
\end{enumerate}

Ce qui donne les tableaux itérés suivants:
\begin{enumerate}
\item
\begin{table}[H]
\centering
\begin{tabular}{c c}

\begin{tabular}{|c|c|c|}
\hline
\textbf{} & \textbf{F1} & \textbf{F2} \\
\hline
\textbf{M1} & 0.5 & 1 \\
\textbf{M2} & 1 & 2 \\
\textbf{M3} & 2 & 1 \\
\hline
\end{tabular}
&
\begin{tabular}{|c|c|c|}
\hline
\textbf{} & \textbf{F1} & \textbf{F2} \\
\hline
\textbf{D1} & 0 & 0 \\
\textbf{D2} & 1 & 3 \\
\hline
\end{tabular}

\end{tabular}
\end{table}

\item
\begin{table}[H]
\centering
\begin{tabular}{c c}

\begin{tabular}{|c|c|c|}
\hline
\textbf{} & \textbf{F1} & \textbf{F2} \\
\hline
\textbf{M1} & 0 & 1 \\
\textbf{M2} & 1 & 2 \\
\textbf{M3} & 2 & 1 \\
\hline
\end{tabular}
&
\begin{tabular}{|c|c|c|}
\hline
\textbf{} & \textbf{F1} & \textbf{F2} \\
\hline
\textbf{D1} & 0 & 0 \\
\textbf{D2} & 0.5 & 3 \\
\hline
\end{tabular}

\end{tabular}
\end{table}

\item
\begin{table}[h]
\centering
\begin{tabular}{c c}

\begin{tabular}{|c|c|c|}
\hline
\textbf{} & \textbf{F1} & \textbf{F2} \\
\hline
\textbf{M1} & 0 & 0 \\
\textbf{M2} & 1 & 2 \\
\textbf{M3} & 2 & 1 \\
\hline
\end{tabular}
&
\begin{tabular}{|c|c|c|}
\hline
\textbf{} & \textbf{F1} & \textbf{F2} \\
\hline
\textbf{D1} & 0 & 0 \\
\textbf{D2} & 0.5 & 3 \\
\hline
\end{tabular}

\end{tabular}
\end{table}

\item
\begin{table}[H]
\centering
\begin{tabular}{c c}

\begin{tabular}{|c|c|c|}
\hline
\textbf{} & \textbf{F1} & \textbf{F2} \\
\hline
\textbf{M1} & 0 & 0 \\
\textbf{M2} & 0.5 & 2 \\
\textbf{M3} & 2 & 1 \\
\hline
\end{tabular}
&
\begin{tabular}{|c|c|c|}
\hline
\textbf{} & \textbf{F1} & \textbf{F2} \\
\hline
\textbf{D1} & 0 & 0 \\
\textbf{D2} & 0 & 3 \\
\hline
\end{tabular}

\end{tabular}
\end{table}

\item
\begin{table}[H]
\centering
\begin{tabular}{c c}

\begin{tabular}{|c|c|c|}
\hline
\textbf{} & \textbf{F1} & \textbf{F2} \\
\hline
\textbf{M1} & 0 & 0 \\
\textbf{M2} & 0.5 & 0 \\
\textbf{M3} & 2 & 1 \\
\hline
\end{tabular}
&
\begin{tabular}{|c|c|c|}
\hline
\textbf{} & \textbf{F1} & \textbf{F2} \\
\hline
\textbf{D1} & 0 & 0 \\
\textbf{D2} & 0 & 1 \\
\hline
\end{tabular}

\end{tabular}
\end{table}

\item
\begin{table}[H]
\centering
\begin{tabular}{c c}

\begin{tabular}{|c|c|c|}
\hline
\textbf{} & \textbf{F1} & \textbf{F2} \\
\hline
\textbf{M1} & 0 & 0 \\
\textbf{M2} & 0.5 & 0 \\
\textbf{M3} & 2 & 0 \\
\hline
\end{tabular}
&
\begin{tabular}{|c|c|c|}
\hline
\textbf{} & \textbf{F1} & \textbf{F2} \\
\hline
\textbf{D1} & 0 & 0 \\
\textbf{D2} & 0 & 0 \\
\hline
\end{tabular}

\end{tabular}
\end{table}
\end{enumerate}

Cela montre que la solution est cohérente étant donné que chaque état de la solution respecte les contraintes données.



 \newpage
\section {Comparaison des calculs de bornes}
En moyenne, l'écart relatif entre le temps de calcul utilisant la borne 1 et le temps de calcul utilisant la borne 2 est d'environ 0.001\%.
Cet écart s'explique par le fait que le calcul de la borne 2 trie directement les objets par ratio de prix sur poids décroissant, ce qui permet de limiter les branchement aux objets les plus pertinents plutôt que de brancher sur celui qui a le ratio maximal. \newpage
\section{Améliorations possibles}
Voici les amélioratons que nous pourrions implémenter:
\begin{itemize}
    \item La méthode de résolution de la relaxation pour la borne 2 est naïve en $O(n)$ avec $n$ la taille de la liste des objets.
    On peut optimiser cela en
    \item Notre règle d'exploration parcourt les noeuds par priorité de borne supérieure et parmis les noeuds à parcourir, on les parcours selon l'ordre décroissant de profondeur de l'arbre.
     C'est une caractéristique propre au fait que notre programme résout récursivement le problème .Pour changer cela il faudrait mettre en place une file, ce qui est moins naturel dans notre cas.
     \item Notre algorithme une structure de donnée ce qui est couteux en mémoire. On pourrait économiser du temps de calcul et de la mémoire en modifiant notre algorithme afin d'en retirer la structure de données.
\end
 \newpage

\end{document}
